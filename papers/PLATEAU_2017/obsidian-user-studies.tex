%% For double-blind review submission
\documentclass[sigplan,10pt,review]{acmart}\settopmatter{printfolios=true}
%% For single-blind review submission
%\documentclass[sigplan,10pt,review]{acmart}\settopmatter{printfolios=true}
%% For final camera-ready submission
%\documentclass[sigplan,10pt]{acmart}\settopmatter{}

%% Note: Authors migrating a paper from traditional SIGPLAN
%% proceedings format to PACMPL format should change 'sigplan' to
%% 'acmsmall'.

%% Some recommended packages.
\usepackage{booktabs}   %% For formal tables:
                        %% http://ctan.org/pkg/booktabs
\usepackage{subcaption} %% For complex figures with subfigures/subcaptions
                        %% http://ctan.org/pkg/subcaption
\usepackage{listings}
\usepackage{mathptmx}

\makeatletter\if@ACM@journal\makeatother
%% Journal information (used by PACMPL format)
%% Supplied to authors by publisher for camera-ready submission
\acmJournal{PACMPL}
\acmVolume{1}
\acmNumber{1}
\acmArticle{1}
\acmYear{2017}
\acmMonth{1}
\acmDOI{10.1145/nnnnnnn.nnnnninn}
\startPage{1}
\else\makeatother
%% Conference information (used by SIGPLAN proceedings format)
%% Supplied to authors by publisher for camera-ready submission
\acmConference[PLATEAU'17]{8th Workshop on Evaluation and Usability of Programming Languages and Tools}{October 23, 2017}{Vancouver, BC, Canada}
\acmYear{2017}
\acmISBN{978-x-xxxx-xxxx-x/YY/MM}
\acmDOI{10.1145/nnnnnnn.nnnnnnn}
\startPage{1}
\fi

\lstset{
  tabsize=3,
  basicstyle={\small\ttfamily},
  stepnumber=1
}

\renewcommand{\lstlistingname}{Fig.}

%% Copyright information
%% Supplied to authors (based on authors' rights management selection;
%% see authors.acm.org) by publisher for camera-ready submission
\setcopyright{none}             %% For review submission
%\setcopyright{acmcopyright}
%\setcopyright{acmlicensed}
%\setcopyright{rightsretained}
%\copyrightyear{2017}           %% If different from \acmYear


%% Bibliography style
\bibliographystyle{ACM-Reference-Format}
%% Citation style
%% Note: author/year citations are required for papers published as an
%% issue of PACMPL.
%style{acmauthoryear}  %% For author/year citations
%\citestyle{acmnumeric}     %% For numeric citations
%\setcitestyle{nosort}      %% With 'acmnumeric', to disable automatic
                            %% sorting of references within a single citation;
                            %% e.g., \cite{Smith99,Carpenter05,Baker12}
                            %% rendered as [14,5,2] rather than [2,5,14].
%\setcitesyle{nocompress}   %% With 'acmnumeric', to disable automatic
                            %% compression of sequential references within a
                            %% single citation;
                            %% e.g., \cite{Baker12,Baker14,Baker16}
                            %% rendered as [2,3,4] rather than [2-4].


\begin{document}

%% Title information
\title{A User Study to Inform the Design of the Obsidian Blockchain DSL}         %% [Short Title] is optional;
                                        %% when present, will be used in
                                        %% header instead of Full Title.


%% Author information
%% Contents and number of authors suppressed with 'anonymous'.
%% Each author should be introduced by \author, followed by
%% \authornote (optional), \orcid (optional), \affiliation, and
%% \email.
%% An author may have multiple affiliations and/or emails; repeat the
%% appropriate command.
%% Many elements are not rendered, but should be provided for metadata
%% extraction tools.

%% Author with single affiliation.
\author{Celeste Barnaby}
\affiliation{
  \institution{Wesleyan University}            
}
\email{cbarnaby@wesleyan.edu}          

\author{Michael Coblenz}
\affiliation{
  \institution{Carnegie Mellon University}            
}
\email{mcoblenz@cs.cmu.edu}          

\author{Tyler Etzel}
\affiliation{
  \institution{Cornell University}            
 }
\email{tje44@cornell.edu} 

\author{Eliezer Kanal}
\affiliation{
  \institution{Carnegie Mellon University}            
}
\email{ekanal@cert.org}

\author{Joshua Sunshine}
\affiliation{
  \institution{Carnegie Mellon University}            
}
\email{sunshine@cs.cmu.edu}

\author{Brad Myers}
\affiliation{
  \institution{Carnegie Mellon University}            
}
\email{bam@cs.cmu.edu} 

\author{Jonathan Aldrich}
\affiliation{
  \institution{Carnegie Mellon University}            }
\email{jonathan.aldrich@cs.cmu.edu}   

%% Paper note
%% The \thanks command may be used to create a "paper note" ---
%% similar to a title note or an author note, but not explicitly
%% associated with a particular element.  It will appear immediately
%% above the permission/copyright statement.
\thanks{with paper note}                %% \thanks is optional
                                        %% can be repeated if necesary
                                        %% contents suppressed with 'anonymous'


%% Abstract
%% Note: \begin{abstract}...\end{abstract} environment must come
%% before \maketitle command
\begin{abstract}

Blockchain platforms such as Ethereum and Hyperledger facilitate transactions between parties that 
have not established trust. Increased interest in these platforms has motivated the design of programming languages
such as Solidity, which allow users to create blockchain programs. However, there have been several recent instances
where Solidity programs have contained bugs that have been exploited. The security of blockchain programs is especially
important given that they commonly involve the exchange of money or other objects with real-world value. 
We are currently developing a blockchain-based programming language called Obsidian with the goal
of minimizing the risk of common security vulnerabilities. We are designing this
language in a human-centered way, conducting exploratory user studies with a natural programming approach to
inform our design choices. In this paper, we discuss our approach to the design of a user study, as well
as our preliminary findings.

\end{abstract}

%% Keywords
%% comma separated list
\keywords{blockchain programming, blockchain security, user study}  %% \keywords is optional


%% \maketitle
%% Note: \maketitle command must come after title commands, author
%% commands, abstract environment, Computing Classification System
%% environment and commands, and keywords command.
\maketitle
\renewcommand{\shortauthors}{Barnaby et al.}

\section{Introduction}

We are designing a blockchain-based programming language called Obsidian \cite{Coblenz} with the goal of 
minimizing the risk of common security vulnerabilities in blockchain programs. Blockchain
programs written with current domain-specific languages such as Solidity \cite{Solidity} often contain
exploitable bugs. In a particularly calamitous example, \$50 million was stolen from a contract called \textit{The DAO} on the 
Ethereum blockchain \cite{Sirer}. Obsidian contains core features
 -- namely, first-class typestate, linear resources, and path-dependent types -- that will allow users 
to write safe, effective programs. The target 
user base of Obsidian is business professionals who will use the language to write smart contracts, 
and its design is thus oriented towards this domain. Obsidian programs consist of contracts --
similar to classes in Java -- which contain fields, states, and transactions -- similar to methods.

\subsection{Typestate}

We (as well as other researchers) have observed that programs in the domains of focus for blockchain platforms are typically state-oriented \cite{State}. Furthermore, the DAO exploit stemmed from invoking a function in an external contract while the calling contract was in an inconsistent state.
In light of this, Obsidian makes state first-class: an object in Obsidian has a mutable state that restricts which 
transactions can be invoked on it \cite{Aldrich}. 

\begin{lstlisting}[caption={An example of states in Obsidian.}, captionpos = b, label = librarycard]
contract LibraryCard {
   state NoCard {
       transaction getCard() {
           ...
           ->HasCard
       }
   }
  
   state HasCard {
       transaction checkOutBook() {
           ...
       }
   }
}
\end{lstlisting}

The states of a contract are defined as explicit blocks containing 
transactions and fields, which can only be accessed if the contract is in the corresponding state. 
In the example in Fig. \ref{librarycard}, a \texttt{\small{LibraryCard}} is always in one of two
states, \texttt{\small{NoCard}} or \texttt{\small{HasCard}}. The transaction 
\texttt{\small{checkOutBook}} can only be called on an instance of 
\texttt{\small{LibraryCard}} if it is in the \texttt{\small{HasCard}} state; otherwise, it will throw an exception. 
The \texttt{\small{->}} operator indicates a state transition.

\subsection{Linear Resources}

Blockchain programs may manage some kind of asset, like a cryptocurrency, or a token
indicating a more complicated right (e.g. ownership of a financial option). Linear types \cite{Wadler} are an existing approach that allow the compiler to
enforce a safe, clean programming model: resources cannot be used more than once, but must be 
used before leaving the current scope (thus ensuring that a asset is not lost accidentally).

\begin{lstlisting}[caption={Linear resources in Obsidian}, captionpos = b, label = treasury]
contract Treasury {
   // Money is a asset of Treasury
   asset contract Money { ... }

    transaction t1(Money m) {
        spendMoney(m);
        Bond b = exchangeForBonds(m);
        // compiler error: m is used twice
    }

    transaction t2(Money m) {
        // compiler error: m is never used
        return;
    }
}
\end{lstlisting}

\subsection{Path-Dependent Types}

There are certain cases in blockchain programs in which a programmer may want a asset to
be dependent upon a specific contract. For example, we may want each instance of the 
\texttt{\small{Treasury}} contract in Fig. \ref{treasury} to mint its own kind of money. If money \texttt{\small{m1}} from treasury \texttt{\small{t1}} is used with a distinct treasury \texttt{\small{t2}}, we would like the compiler to give an error message. Obsidian supports this via the inclusion of path-dependent types \cite{Amin}, wherein values can have type members. In the 
above example, the type of Money is dependent upon a specific value of Treasury; put another way, 
each treasury \texttt{\small{t}} has its own type \texttt{\small{t.Money}}. 

Path-dependent types assist in writing secure contracts: without such types, every
treasury would share the same money type, and the programmer would have to manually check 
that all money deposited into a treasury (for example) is of the correct type. Incorrect or insufficient 
checks could leave contracts vulnerable to exploitation. With path-dependent types, all such checks 
are automatic: we can ensure that only money that comes from a specific treasury is used in 
transactions within that treasury. 

\subsection{Usability}
It is crucial that real programmers are able to write \textit{correct} Obsidian 
code easily and efficiently. But even with seemingly intuitive features, it is not always clear 
which design is most effective for programmers. For instance, consider the following simple 
Obsidian contract:

\begin{lstlisting}
contract C {
   state Start {
       int x;
  
       transaction a() returns int {
           ->S1{x1 = x};
           return b(x1);
       }
   }
  
   state S1 {
       int x1;
      
       transaction b(int y) returns int {
           return y;
       }
   }
}
\end{lstlisting}

In transaction \texttt{\small{a}}, does it make sense to call \texttt{\small{b(x1)}} after transitioning 
to state \texttt{\small{S1}}? Lexically the contract is still in the 
\texttt{\small{Start}} state, so it is not immediately apparent which variables and transactions are in scope. 
In a nontrivial program with many different contracts, 
transactions, and states, a situation like this could cause serious confusion. Failure to encode state 
machines properly has been shown to be a significant source of errors in smart-contract 
programming \cite{Delmolino}. It is critical that users are able to understand and use states easily and 
effectively -- if not, there is potential to create the same or worse bugs, including security weaknesses, as in a 
language without states. 

Similar questions of usability arise with linear resources and path-dependent types. 
Despite their apparent utility, linear types have seen limited adoption in popular programming languages (though 
Rust uses a form of linearity for alias control \cite{Rust}), so it is not clear what is the best user-facing
approach to integrating these types into our design. As for path-dependent types, an initial approach we have 
taken is to use nested contracts to indicate a dependency relationship; however, nesting has different 
implications in different languages, and we are not certain that users will be able to recognize the 
presence and utility of path dependency. 

A related issue is whether users will want to use these features if they are made available. Are 
these logical, sensible solutions to real problems that programmers face? If people do not 
understand or choose to use the language's core features, then any of the potential benefits 
of those features are lost. 

We are conducting a user study to investigate the usability of state transitions and path-dependent types in 
Obsidian (a study of the usability of linear resources will occur in future work). This study is consistent with 
prior work that showed the applicability of human-computer interaction techniques to programming tools 
\cite{Myers}. One technique we use is the \textit{natural programming} approach, in which we ask participants 
how they would like to express their solutions to programming problems \cite{Natural}. The key research 
questions we address in the study are as follows: 
\begin{itemize}
\item Are states and path-dependent types a natural way of 
	approaching the challenges that arise in blockchain programming?
\item How do people naturally express states, state transitions, and path-dependent types?
\item Which (if any) of our proposed ways of presenting states, state transitions, and path-dependent types
	is most understandable and usable by programmers?
\item Are people able to effectively use and understand path-dependent types as they
are implemented in Obsidian?
\end{itemize}

\section{User Study Design}
	
This study was exploratory rather than evaluative: its purpose was to give us information about the usability of state transitions and path-dependent types so we can make informed choices about the design of our language. 
We chose these features as the focus of our study because they are both key safety features 
that we expect users will use frequently. In addition, they both have several distinct options for their syntactic representation, and the results of the study will factor into which design we choose to implement.  

Participants were asked to complete two programming exercises, both of which were divided into parts that 
gradually introduced the participant to Obsidian and its main features. 
Participants were instructed to think aloud throughout the exercise, and were permitted to ask questions. We designed the study so that both exercises could be completed in approximately an hour and a half, in order to make it easier for us to find willing participants as well as to limit participant fatigue. We obtained IRB approval for the study; participants gave informed consent and were paid \$10/hour for their time.

\subsection{Voter Registration Exercise}

Participants were given a description of a voter registration system for a hypothetical democratic nation. The 
system had certain stipulations that made a state machine a logical means of representing its required behaviors. 
For instance, the system had specific conditions under which a citizen either became registered to vote 
or remained unregistered. 

The exercise was divided into five parts. In part \textbf{one}, participants were asked to implement the system using 
pseudocode. They were encouraged to invent any language features that they wanted in order to solve this 
problem. Our goal was to see how people naturally solve a problem in this domain: what ideas do they 
have, and what assumptions do they make? 
	
In part \textbf{two}, participants were given a state diagram that modeled the voter registration system, and were asked to 
modify their pseudocode to include the states and state transitions shown in the diagram. Again, we wanted to 
observe people's natural ideas about how to represent states and state transitions in a program.

In part \textbf{three}, participants were given a two-page Obsidian tutorial detailing the key components of the language. 
The tutorial explained how state blocks work, but did not give any information about how transitions should be 
written. Participants were then given an Obsidian program that implemented the voter registration system, but was 
missing state transitions. Participants were asked to add state transitions to the code, inventing the syntax 
themselves.

In part \textbf{four}, participants were shown three options for the syntax and functionality of state transitions, each 
accompanied by a short code example. Participants were presented the options in a random order; the order given 
here is arbitrary.

In option 1, shown in Fig. \ref{transitions1}, users were allowed to use any transaction available in the current \textit{dynamic} state regardless of the lexical context. For instance, it is legal to use the \texttt{\small{toS2}} transaction (on line 5) inside the \texttt{\small{Start}} 
state, even though that transaction is defined within \texttt{\small{S1}}. This is because there is a transition to 
\texttt{\small{S1}} in the previous line.


\begin{lstlisting}[caption={Option 1 for state transitions},captionpos=b, label=transitions1, numbers=left, xleftmargin=.5cm]
contract C {
   state Start {
       transaction t(int x) {
           ->S1{x1 = x};
           toS2();
       }
   }

   state S1 {
       int x1;

       transaction toS2() {
           ->S2{x2 = x1};
       }
   }

   state S2 {
       int x2;
   }
}
\end{lstlisting}

In option 2, shown in Fig. \ref{transitions2}, each state had a constructor that was invoked when the contract transitioned to that state. 
With this option, there could not be any code following a state transition; thus, a transition had to be the final line of a 
transaction. 

\begin{lstlisting}[caption = {Option 2 for state transitions}, captionpos = b, label = transitions2, numbers = left,
xleftmargin = .5cm]
contract C {
    state Start {
        transaction t(int x) {
            ->S1(x)
        }
    }

    state S1 {
        int x1;
        S1(int x) { // State constructor
            x1 = x;
            ->S2(x1);
        }
    }

    state S2 {
        int x2;
        S2(int x) { // State constructor 
            x2 = x;
        }
    }
}
\end{lstlisting}

In option 3, shown in Fig. \ref{transitions3}, there were conditional \texttt{\small{if in \{state\}}} blocks, which allowed the user
to lexically nest states so that another state's transactions and fields could be used directly. 

\begin{lstlisting}[caption = {Option 3 for state transitions}, captionpos = b, label = transitions3, numbers = left,
xleftmargin = .5cm]
contract C {
    state Start {
        transaction t(int x) {
            ->S1({x1 = x})
            if in S1 {
                ->S2({x2 = x1})
            }
            if in S2 {
            	...
            }
        }
    }

    state S1 {
        int x1;
    }

    state S2 {
        int x2;
    }
}
\end{lstlisting}

Participants were asked to complete a short Obsidian contract once for each option. The contract was 
designed to be a simple yet non-trivial use of state transitions that illustrated the benefits and drawbacks of each 
option. The goal of this part was to see whether participants would be able able to implement the 
contract successfully with each option, as well as to gather feedback about which option they preferred and why.

In part \textbf{five}, participants were asked to pick one of the three options for state transitions and use it to complete 
the Obsidian program from part three. They were then asked to explain their reasoning and elaborate on if 
there was anything confusing about any of the options. 

\subsection{Lottery Ticket Exercise}
	
Participants were given a description of a program that allowed users to create and participate in lotteries. Every 
lottery sold lottery tickets, but a user should only be able to redeem a winning ticket from the lottery from which it was purchased
 -- thus motivating the use of path-dependent types. 

The exercise was divided into two parts. Part one mirrored the voter registration exercise in that participants were 
asked to implement the program using pseudocode. Again, we wanted to see how people naturally go about 
solving this problem. Would using path-dependent types -- or some feature similar to that -- occur to anyone?
	
In part two, participants were given an explanation of path-dependent types and offered an Obsidian contract that 
implemented the lottery program, but had two transactions left unwritten. Participants were asked to write 
those transactions. We wanted to observe whether people were able to understand path-dependent types and write 
correct code using them after only a brief introduction. 

\section{Discussion of Study Design}

One challenge in designing the user study was ensuring that the programming exercises had an appropriate level of difficulty. The 
scenarios had to be simple enough that participants could comprehend and implement them in the little time they had, but 
 complex enough that implementing them was a non-trivial problem that actually motivated the use of 
Obsidian's features. Our first several pilot studies revealed that our exercises were too complicated, and 
participants took much longer to read and understand the instructions than we had anticipated. Additionally, there were parts 
of the exercises that people were continually confused about, which made it difficult to assess their ability to use the language. 

As we revised the programming exercises, we trended towards simplifying and condensing. For instance, the state diagram in 
the voter registration exercise originally had six states and five transitions, but was modified to have three states and three 
transitions; the tutorial was cut from three pages to one and a half; and two parts were removed from the lottery ticket exercise. 
We also made sure that each exercise targeted exactly one feature: the voter registration exercise was focused only on state 
transitions, the lottery ticket exercise on path-dependent types. 

We found that simplifying the exercises allowed us to collect better data. Participants who completed the simplified exercises 
spoke their thoughts aloud more consistently and stated their preferences and opinions more confidently. They were able
come up with better and more interesting solutions to the problems and write Obsidian code more effectively. 
But simplifying our programming exercises also created certain limitations. Since the exercises were short and not very 
complex, participants' opinions may have been based only on a cursory understanding of the language. There may be an option 
for state transitions, for instance, whose utility only becomes apparent in a large, complicated contract. 
Testing these issues is left for future work.

\section{Preliminary Results}

We recruited a convenience sample of 12 participants. They had varying levels of programming experience: some were beginners, some 
experts. None had any knowledge about Obsidian prior to completing the study. Nine of the participants were undergraduates 
studying computer science; one was a computer science Ph.D student, and two were working in a business-related fields. Since this was an exploratory study, we revised the study materials after each participant according to what we learned about the materials or the language design choices, and we asked participants to complete either one or both exercises according to our experimental design needs and the participants' time constraints. Some participants only completed several parts of one exercise due to time constraints. 

\subsection{Voter Registration Exercise}

Seven participants were given the voter registration exercise. When asked to write pseudocode for the voter registration exercise, the 
general approach every participant took was to create a globally accessible list that stored the registration status (either registered or 
unregistered) of each citizen. While this is a logical implementation of the problem, it was not completely secure. For example, some 
participants created separate lists for registered and unregistered citizens, meaning that it would be possible for a citizen to erroneously 
appear on both lists. 
 	
Six out of these seven participants were shown a state diagram and asked to modify their pseudocode to use states. Of these six 
participants, two created explicit state blocks with functions and variables inside, similar to the design of Obsidian. The rest either maintained a global state variable that 
changed based on the status of a citizen, gave each citizen a state field that changed based on the citizen's status (e.g. with syntax such as 
"\texttt{\small{Citizen.state = CANVOTE()}}"), or created empty, immutable states at the top of the program. Several participants did not check whether a 
citizen was unregistered before processing their application, meaning it would be possible for an already registered citizen to register again 
-- something we expressly prohibited in the instructions. 
	
When looking at and writing Obsidian code with states, participants asked a lot of questions about what should be allowed to happen 
during and after a state transition -- that is, what variables are and are not in scope, what the keyword ``\texttt{\small{this}}" refers to, and what transactions 
can be used. Several participants asked if there was any way to check which state the contract was in. One participant noted that he felt it 
should never be allowed to call transitions from one state while lexically in another, saying ``I'm calling S1's transaction from code for Start.''
Another participant said that she felt that state transitions were like return statements, and after completing a transition there should not be 
any more code in that transaction.

Three participants preferred the option that included state constructors, maintaining that this option was easier to understand. One preferred the option with \texttt{\small{if in \{state\}}} blocks because it made it immediately apparent which state a contract was in. The remaining three participants either did not express a preference or did not complete this part of the exercise. 

\subsection{Lottery Ticket Exercise}	
	
Six participants were given the lottery ticket exercise. When asked to implement the program using pseudocode, four out of the six defined 
a class for Lottery. The instructions specified that users of the program should be able to buy a ticket from any lottery, but must only be 
able to redeem a winning ticket from the lottery where they bought the ticket. Four out of the six implemented a program without 
immediately recognizing or forming a solution to this problem. When the study facilitator pointed out the issue, the approach all four 
participants took to resolve it was to give every lottery a fixed ID. They then made sure that the function that redeems a ticket must check 
that the ticket's ID is equal to the lottery ID. 

This implementation left some room for exploitation. Two participants made the lottery's ID a randomly generated number, meaning it would be 
possible for two lotteries to have the same ID. One participant had the ticket owner input the lottery ID themselves upon redeeming 
the ticket, meaning that if a ticket owner somehow found the ID of a different lottery, they could redeem their ticket from there.
In each case, the participant was able to understand the need to have lottery tickets be tied to 
lotteries in some way, but four out of the six participants had trouble executing this easily and effectively. 

When offered an explanation of path-dependent types and given an Obsidian program to complete, all participants were able to write 
correct (albeit very simple) code. Five participants were asked to identify the types of two lottery tickets that had been purchased from different 
lotteries. Since lotteries and lottery tickets had an established dependency relationship, the correct answer was that they had different 
types, even though they were both lottery tickets. Of these five participants, two were able to offer this answer with accurate reasoning. 
One vaguely said that it "seems" like they should have different types, but was not sure since it was not indicated explicitly in the code. 

Three participants noted that the use of nested classes was confusing or unclear -- one said, "it's usually bad practice to use nested 
classes in Java."

\section{Discussion}

The results of the pseudocode portions of both exercises indicate that states and path-dependent types are not necessarily the most obvious or 
natural ways of solving the problems we presented. This makes sense given the backgrounds of our participants: most of them noted that they 
were writing pseudocode resembling the language they were most comfortable with, and thus may not have thought about inventing new, 
unfamiliar language features. Moreover, the simplicity of both exercises -- the voter registration exercise in particular -- may have made the 
need for these features somewhat opaque. Still, we found it encouraging that two participants independently invented special syntax denoting states, with appropriate scoping for fields and transactions. Further, we found in many cases that the approaches participants did take (reflecting approaches representative of commonly-used languages) to implementing the 
programs were insufficient or unsafe, and the weaknesses in their programs could have been solved by using states or path-dependent 
types. This result, coupled with our strong background evidence of the utility of these features, motivates us to continue developing these
features in Obsidian, while further investigating the best ways to design, present, and evaluate them. 

The results of the voter registration exercise indicate that a majority of participants prefer to specify code that executes after state transitions using state constructors. The 
fact that participants preferred this option after a short coding exercise is not conclusive proof that this is the best option or the one we should 
implement; however, it does indicate that Obsidian users want it to be simple and easy to tell which variables and transactions are in scope and to lexically determine the 
current state of an object. Our results offer evidence that encapsulating all the actions of a state within that state may allow users to understand 
more easily which state an object is currently in and which transactions and fields they are allowed to use -- thus enabling them to write better code. 

The responses we received from participants in the lottery ticket exercise reveal that nesting contracts is likely not the most understandable
way to express a dependency relationship. Three participants made comments about this, and those who did not were not 
able to identify path-dependent types correctly. An alternative to this approach would be to prohibit nesting and instead use a keyword
to denote this relationship (e.g. \texttt{\small{"asset contract LotteryTicket depends on Lottery"}}).

\section{Future Work}

We are continuing to refine the language features and test them with further user studies.
\begin{itemize}
\item We will target people in the business domain (e.g. business students and business analysts) with limited
programming experience, in order to collect data from the intended user base of Obsidian.
\item We will design programming exercises that further address the usability of linear resources. One question
of interest, for example, is how to enforce linearity for field accesses and state transitions: what should happen
when attempting to access an owned field that has already been consumed, and how can one transition
between states with different owned fields? 
\item We will design programming exercises that require participants to read and write longer, more complicated 
Obsidian contracts that actually compile. This will offer us more evidence about whether people are able to write 
correct Obsidian code. It will also allow participants to gain a deeper, less superficial understanding of Obsidian's 
features and thus offer more constructive feedback about Obsidian's usability. 
\item Finally, we plan to modify the Obsidian language implementation using the results of these studies, 
and evaluate the final design in a formal study testing its effectiveness.
\end{itemize}

\section{Conclusion}

We designed and conducted an exploratory study of the usability of two of Obsidian's major safety features. 
Preliminary results from this study offered valuable insight into both the design choices we will make in the 
language as well as the direction that future user studies will take. By using a human-centered approach in the 
design of Obsidian, we aim to offer a blockchain-based programming language that allows users to write smart 
contracts more safely and easily than currently available blockchain DSLs.


%% Acknowledgments
%\begin{acks}                            %% acks environment is optional
                                        %% contents suppressed with 'anonymous'
  %% Commands \grantsponsor{<sponsorID>}{<name>}{<url>} and
  %% \grantnum[<url>]{<sponsorID>}{<number>} should be used to
  %% acknowledge financial support and will be used by metadata
  %% extraction tools.
  %This work primarily took place while Celeste and Tyler were in an REU program at CMU, 
  %funded by NSF grant CNS-1734138. Other support comes from CNS-1423054, and ****<anything for anyone
  %else?****>
 %\end{acks}


%% Bibliography
\begin{thebibliography}{9}
\bibitem{Coblenz}
	M. Coblenz, ``Obsidian: A Safer Blockchain Programming Language," 
	in \emph{Proceedings of the 39th International Conference on Software Engineering -
	ICSE '17,} 2017.
\bibitem{Solidity}
	Ethereum Foundation, ``Solidity," https://solidity.readthedocs.io/en/develop/. Accessed Aug. 3, 2017.
\bibitem{Sirer}
	E. G{\"u}n Sirer, ``Thoughts on the DAO hack," 2016. [Online]. 
	Available: http://hackingdistributed.com/2016/06/17/thoughts-on-the-dao-hack/
\bibitem{Aldrich}
	J. Aldrich, J. Sunshine, D. Saini, Z. Sparks,
	``Typestate-Oriented Programming,"
	in \emph{Proceedings of the 24th ACM SIGPLAN conference companion on Object oriented 
	programming systems languages and applications,} 2009, pp. 1015-1022.
\bibitem{Wadler}
	P. Wadler, ``Linear Types Can Change the World,"
	IFIP TC, vol. 2, pp. 347 - 359, 1990.
\bibitem{Amin}
	 N. Amin, T. Rompf, and M. Odersky. ``Foundations of Path-Dependent Types."
	 in OOPSLA, 2014.
\bibitem{Delmolino}
	K. Delmolino, M. Arnett, A. E. Kosba, A. Miller, and E. Shi, 
	``Step by step towards creating a safe smart contract: Lessons and insights from a cryptocurrency lab."
	IACR Cryptology ePrint Archive, vol. 2015, p. 460, 2015.
\bibitem{Myers}
	B. Myers, A. Ko, T. LaToza, and Y. Yoon, 
	``Programmers Are Users Too: Human-Centered Methods for Improving Programming Tools,"
	IEEE Computer, Special issue on UI Design, 49, issue 7, July, 2016, pp. 44-52.
\bibitem{State}
	Ethereum Foundation, ``Common patterns,''
	http://solidity.readthedocs.io/en/develop/common-patterns.html.
	Accessed Jan. 4, 2017.
\bibitem{Rust}
	N. D. Matsakis and F. S. Klock, II. 2014. ``The Rust language." Ada Lett. 34, 3 (October 2014), 103-104.
\bibitem{Natural}
	B.A. Myers, J.F. Pane, A. Ko, ``Natural Programming Languages and Environments", Comm. ACM, vol. 47, no. 9, pp. 47-52, 2004.	
\end{thebibliography}



\end{document}
